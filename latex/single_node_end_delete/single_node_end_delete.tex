\documentclass{article}
\usepackage[noend]{algpseudocode}
\usepackage{amsmath}
\usepackage{amsthm}
\usepackage{amssymb}

\author{Auburn University - Software Construction \\ Austin Chase Minor}
\title{Single Linked List End Node Deletion Using One
  Pointer}
\date{\today}

\renewcommand{\qedsymbol}{$\spadesuit$}

\begin{document}
\maketitle

\begin{algorithmic}[1]
  \Function{delete\_last\_node}{Node* list}
  \If {list == null}
  \State \Return 0
  \EndIf
  \If {list-\textgreater next == null}
  \State delete list
  \State list = null
  \State \Return 0
  \EndIf
  \While {list-\textgreater next-\textgreater next != null}
  \State list = list-\textgreater next
  \EndWhile
  \State delete list-\textgreater next
  \State list-\textgreater next = null
  \State \Return 0
  \EndFunction
\end{algorithmic}

\text{}\\
\textbf{Proof}
\begin{description}
\item[Loop Invarient]
  \text{}\\
  At the start of the while loop on line 8, list.next is guarenteed to
  exist.
\item[Initialization]
  \text{}
 
 If the loop is reached in line 8, we know that (i) list is not null and (ii)
 list.next is not null. (i) follows from lines 2-3 and (ii) follows from lines
 4 - 7. Thus for (i) nothing is deleted. Thus for (ii) the last node (which
 happens to be the first node) is deleted. Thus the loop invarient holds
 before the first time the loop is executed.
\item[Maintenance]
  \text{}

  The loop has to possibilities. Either (i) list.next.next does not exist
  (null) or (ii) list.next.next does exist. In (i), the loop breaks and we
  know from the loop invarient that list.next exists (since no work was
  done). In (ii), list.next.next does exist. Thus line 9 executes setting list
  to list.next. Since list.next.next exists and list = list.next we know that
  the new list.next must exist. Thus the loop holds for maintenace.
\item[Termination]
  \text{}

  Since the loop invarient holds for maintenance, we know that at the end of
  the loop that (i) list.next exists and (ii) list.next.next does not
  exist. Thus, list.next is the last element. This element is thus deleted and
  set to null. Furthermore, it can be easily shown that this strategy applies
  to circular lists. Just replace line 11 with list.next = head\_node and
  replace the check in line 8 with list.next.next != head\_node. Therefore,
  this algorithm is correct.
\end{description}
\textbf{Efficiency}\\

For comparison purposes presented bellow is the algorithm for a different two
node version of delete\_last\_node modified from Dr. Qin C++ lecture slides.\\

\begin{algorithmic}[1]
  \Function{delete\_last\_node}{Node* list}
  \State Node* (cur\_ptr, pre\_ptr) = null
  \If {list == null}
  \State \Return 0
  \EndIf
  \If {list.next == null}
  \State delete list
  \State list = null
  \State \Return 0
  \EndIf
  \State pre\_ptr = list;
  \State cur\_ptr = list-\textgreater next;
  \While {cur\_ptr-\textgreater next != null}
  \State pre\_ptr = cur\_ptr;
  \State cur\_ptr = cur\_ptr-\textgreater next;
  \EndWhile
  \State delete list-\textgreater next
  \State list-\textgreater next = null
  \State \Return 0
  \EndFunction
\end{algorithmic}

\text{}\\

Obviously some simplications were made in the algorithms listed above compared
to real implementations (dealing with pointer references, etc.). However, the
proof and general memory/speed analysis is well approximated.\\
\textbf{Memory Efficiency}\\

Through analysis it is obvious that the single node algorithm presented needs
only space for one pointer. In c++, it would need space for two pointers
because of the pointer reference issue. The double node algorithm in c++ would
require the use of three pointers: list, cur\_ptr, and pre\_ptr. Thus even in
c++, the single node algorithm is more space efficient by one address
variable.\\
\textbf{Speed Efficiency}\\

Establishing constants on every needed line of the single node algorithm and
assuming for simplicity that every if statement executes
yields: $T(N) = c_2 + c_3 + c_4 + c_5 + c_6 + c_7 + (K_N + 1) * c_8 + K_N *
c_9 + c_{10} + c_{11} + c_{12}$ Where $K_N$ is the number of items in the list
less 2. I.E. the algorithm stops on the next, next to last element. Applying
some napkin calculations to the time constants yields: $T(N) = 2 + 1 + 4 + 5 +
2 + 1 +1 + (N-1)*5 + (N-2)*3 + 3 + 3 + 1 = 8N + 12$.

Doing a similar analysis for the two node algorithm yields: $T(N) = 2 + 2 + 1
+ 4 + 2 + 1 + 1 + 2 + 4 + (N-1)*4 + (N-2)*2 + (N-2)*4 + 3 + 3 + 1 = 10N + 8$

Thus using back of the napkin speed efficiency analysis the one node ($T(N) =
8N + 12$)  and two
node ($T(N) = 10N + 8$) algorithms have similar speed efficiency. Thus the one
node algorithm performs similarly to the two node algorithm with a added space efficiency.
\end{document}
